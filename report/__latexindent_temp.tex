@Article{Beer1991,
author="Beer, Tom",
title="The interaction of wind and fire",
journal="Boundary-Layer Meteorology",
year="1991",
month="Feb",
day="01",
volume="54",
number="3",
pages="287--308",
abstract="The rate of spread of a wildfire increases markedly when a wind springs up. Why and how this happens is still not completely understood. However, by using a judicious mixture of laboratory experiments, field experiments, sensitivity analyses of existing wildfire spread models and physical reasoning it is possible to identify some features that have not been adequately considered in the past. In particular: (i) the energetics of wind-blown wildfires indicate that a fire-wind may not exist and that the wind may blow through the fire line. ii) The rate of spread of the fire-front depends on the atmospheric stability such that the fire-front speeds are 50{\%} or more faster during winds in the 2 to 6 m/s range under unstable conditions, iii) The wind speed acting on the flame provides an upper limit to the flame propagation speed. Existing attempts to constrain the propagation rate of fire-spread models at high wind speeds appear incorrect and, iv) wind fluctuations interacting with the standing fuel generate sweeps of downward air which carry the flame into the fuel bed and directly preheat the fuel.",
issn="1573-1472",
doi="10.1007/BF00183958",
url="https://doi.org/10.1007/BF00183958"
}

