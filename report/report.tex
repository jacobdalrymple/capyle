% Author: Thomas Barratt
% Version: 1.0
% This work is licensed under a Creative Commons Attribution 4.0 International License.

\documentclass[11pt, a4paper, titlepage]{article}
% \usepackage[T1]{fontenc}
% \usepackage[latin1]{inputenc}
\usepackage[english]{babel}
\usepackage{siunitx}
\usepackage{graphicx}
\usepackage{tipa} % for the \ark{} command
\usepackage{graphics} % for pdf, bitmapped graphics files
% \usepackage{times} % assumes new font selection scheme installed
\usepackage{amsmath}
\usepackage{latexsym}
\usepackage{amscd}% for commutative diagrams
% \usepackage{mathrsfs} %this package is for the script font \mathscr
\usepackage{relsize}
\usepackage{delarray}
\usepackage{pstricks}
\usepackage{theorem}
\usepackage{changepage}
\usepackage{euscript}
% \usepackage{textcomp}
\usepackage{esvect}
\usepackage{parskip}
\usepackage{placeins}
\usepackage{subfigure}
% \usepackage{subcaption}
\usepackage{array}
\usepackage{delarray}
\usepackage{stmaryrd}
\usepackage{fancyhdr}
\usepackage{graphpap}
\usepackage{makeidx}
\usepackage{enumerate}
\usepackage{esint}
\usepackage{datetime}
\usepackage{caption}
\usepackage{smartdiagram}
\usesmartdiagramlibrary{additions}
%Set Abstract Page
\usepackage{abstract}
\setlength{\absleftindent}{-5mm}
\setlength{\absrightindent}{-5mm}

%Colour definitions - put before TikZ
\usepackage{color}
\definecolor{igreen}{rgb}{0.0, 0.56, 0.0}
\usepackage{xcolor, colortbl}
\colorlet{gred}{-red!75!green!65!}
\colorlet{mamber}{-red!75!green!15!blue!50!}
\colorlet{grown}{-red!75!blue!20!green}
\colorlet{bled}{-red!85!blue!40!green!45!}
\colorlet{waters}{cyan!25} % Define color for the water
\colorlet{water}{cyan!25!green!20!} % Define color for the water
\definecolor{grin}{HTML}{00F9DE}
\usepackage{rotating}
\providecommand{\keywords}[1]{\textbf{\textit{Keywords---}} #1}

% For faint dotted table line
\usepackage{arydshln}
\setlength{\dashlinedash}{.4pt}
\setlength{\dashlinegap}{.8pt}

\usepackage{booktabs}
\usepackage{graphicx}
\usepackage{tikz}
\usepackage{tikz-3dplot}
\usetikzlibrary{
arrows,
arrows.meta,
automata,
backgrounds,
calc,
decorations,
decorations.pathmorphing,
decorations.pathreplacing,
decorations.fractals,
external,
fit,
matrix,
petri,
positioning,
shadows,
shapes,
shapes.multipart,
topaths,
intersections
}
\usepackage{eso-pic}
\def\ba{\begin{array}}
\def\ea{\end{array}}
\def\beann{\begin{eqnarray*}}
\def\eeann{\end{eqnarray*}}
\def\bea{\begin{eqnarray}}
\def\eea{\end{eqnarray}}
\def\bsy{\boldsymbol}
\def\gray#1{{\color{gray}#1}}

%% COUNTERS
\setcounter{MaxMatrixCols}{20}
% \renewcommand{\thesection}{\arabic{section}}
% \renewcommand{\thesection}{\thechapter.\number\numexpr\value{section}}
% \renewcommand{\thesubsection}{\thesection.\number\numexpr\value{subsection}}
%%For changemargin
% \def\quote{\list{}{\rightmargin\leftmargin}\item[]}
% \let\endquote=\endlist 
% \def\changemargin#1#2{\list{}{\rightmargin#2\leftmargin#1}\item[]}
% \let\endchangemargin=\endlist 
% \makeatletter
% \newlength\qvec@height
% \newlength\qvec@depth
% \newlength\qvec@width
% \newcommand{\qvec}[2][]{
%     \settoheight{\qvec@height}{$#2$}
%     \settodepth{\qvec@depth}{$#2$}
%     \settowidth{\qvec@width}{$#2$}
%   \def\qvec@arg{#1}
%   \raisebox{.2ex}{\raisebox{\qvec@height}{\rlap{% 
%     \kern.05em
%     \begin{tikzpicture}[scale=1,shorten >=-3pt,shorten <=-3pt]
%     \pgfsetroundcap
%     \coordinate (Stx) at (.05em,0) ;
% 		\coordinate (Arx) at (\qvec@width-.05em,0) ;
%     \draw[->](Stx) to[bend left] (Arx);
%     \end{tikzpicture}
%   }}}
%   #2
% }
% \makeatother
% \makeatletter
% \newlength\pvec@height
% \newlength\pvec@depth
% \newlength\pvec@width
% \newcommand{\pvec}[2][]{
%     \settoheight{\pvec@height}{$#2$}
%     \settodepth{\pvec@depth}{$#2$}
%     \settowidth{\pvec@width}{$#2$}
%   \def\pvec@arg{#1}
%   \raisebox{.2ex}{\raisebox{\pvec@height}{\rlap{% 
%     \kern.05em
%     \begin{tikzpicture}[scale=1,shorten >=-3pt,shorten <=-3pt]
%     \pgfsetroundcap
%     \coordinate (Stx) at (.05em,0) ;
% 		\coordinate (Arx) at (\pvec@width-.05em,0) ;
%     \draw[->](Stx) to[bend right] (Arx);
%     \end{tikzpicture}
%   }}}
%   #2
% }
% \makeatother
% \makeatletter
% \newlength\vvec@height%
% \newlength\vvec@depth%
% \newlength\vvec@width%
% \newcommand{\vvec}[2][]{%
%   \ifmmode%
%     \settoheight{\vvec@height}{$#2$}%
%     \settodepth{\vvec@depth}{$#2$}%
%     \settowidth{\vvec@width}{$#2$}%
%   \else 
%     \settoheight{\vvec@height}{#2}%
%     \settodepth{\vvec@depth}{#2}%
%     \settowidth{\vvec@width}{#2}%
%   \fi%
%   \def\vvec@arg{#1}%
%   \def\vvec@dd{:}%
%   \def\vvec@d{.}%
%   \raisebox{.2ex}{\raisebox{\vvec@height}{\rlap{%
%     \kern.05em%
%     \begin{tikzpicture}[scale=1]
%     \pgfsetroundcap
%     \draw (.05em,0)--(\vvec@width-.05em,0);
%     \draw (\vvec@width-.05em,0)--(\vvec@width-.15em, .075em);
%     \draw (\vvec@width-.05em,0)--(\vvec@width-.15em,-.075em);
%     \ifx\vvec@arg\vvec@d%
%       \fill(\vvec@width*.45,.5ex) circle (.5pt);%
%     \else\ifx\vvec@arg\vvec@dd%
%       \fill(\vvec@width*.30,.5ex) circle (.5pt);%
%       \fill(\vvec@width*.65,.5ex) circle (.5pt);%
%     \fi\fi%
%     \end{tikzpicture}%
%   }}}%
%   #2%
% }
\makeatother
\def\ba{\begin{array}}
\def\ea{\end{array}}
\def\beann{\begin{eqnarray*}}
\def\eeann{\end{eqnarray*}}
\def\bea{\begin{eqnarray}}
\def\eea{\end{eqnarray}}
\def\bsy{\boldsymbol}
\def\gray#1{{\color{gray}#1}}
% \usepackage{titlesec}
\usepackage{multirow}
%To reference within text
\usepackage{hyperref}
\usepackage{apacite}
\usepackage{lipsum}
\usepackage{tikz-cd}
\usepackage{float}
% \usepackage{titling}
\usepackage{epigraph}
\usepackage[title, titletoc]{appendix}
\setlength\epigraphwidth{8cm}
\setlength\epigraphrule{0pt}

% \titleformat{\chapter}{\normalfont\LARGE}{\thechapter\,$\vert$}{20pt}{\LARGE}{\setcounter{chapter}{0}}
\setlength{\headheight}{15pt}
% \titlespacing*{\chapter}{0pt}{-70pt}{40pt} %Move chapter titles up
% Title page logos:
% \makeatletter
% \newcommand\BackgroundPicturea[3]{
% 	\setlength{\unitlength}{1pt}
% 	\put(0,\strip@pt\paperheight){
% 		\parbox[t]{\paperwidth}{
% 			\vspace{#2}\hspace{#3}
% 			\mbox{\includegraphics[scale=0.5]{#1}}
% }}}
% \newcommand\BackgroundPictureb[3]{
% 	\setlength{\unitlength}{1pt}
% 	\put(0,\strip@pt\paperheight){
% 		\parbox[t]{\paperwidth}{
% 			\vspace{#2}\hspace{#3}
% 			\mbox{\includegraphics[scale=0.3]{#1}}
% }}}
\makeatother
% For my abbreviations
\newcommand{\abbrlabel}[1]{\makebox[3cm][l]{\textbf{#1}\ \dotfill}}
\newenvironment{abbreviations}{\begin{list}{}{\renewcommand{\makelabel}{\abbrlabel}}}{\end{list}}
% Line Spacing
\usepackage{setspace}
\setstretch{1.1}
%Set of command is for the changemargin environment
\def\quote{\list{}{\rightmargin\leftmargin}\item[]}
\let\endquote=\endlist 
\def\changemargin#1#2{\list{}{\rightmargin#2\leftmargin#1}\item[]}
\let\endchangemargin=\endlist
%Replace Contents to Table of Contents	
\addto\captionsenglish{
	\renewcommand{\contentsname}%
	{Table of Contents}
	\setcounter{tocdepth}{3}% Include \subsubsection in ToC
	\setcounter{secnumdepth}{3}% Number \subsubsection in ToC
	}
\renewcommand{\listfigurename}{List of Figures}
\renewcommand{\listtablename}{List of Tables}
\usepackage{geometry}
\usepackage{csquotes}
\usepackage{mathtools}
\usepackage{multicol}
\usepackage{hyperref}
\usepackage{apacite}
\usepackage[title, titletoc]{appendix}
\usepackage[parfill]{parskip}

\geometry{lmargin=37mm, rmargin=25mm} 

% Better inline directory listings
\usepackage{xcolor}
\definecolor{light-gray}{gray}{0.95}
\newcommand{\code}[1]{\colorbox{light-gray}{\texttt{#1}}}
 
% \hypersetup{pdftitle = Applications of Cellular Automata: Wildfire Spread simulation, pdfauthor = {Thomas Barratt, Jacob Dalrymple}, pdfstartview=FitH, pdfkeywords = essay, pdfpagemode=FullScreen, colorlinks, anchorcolor = red, citecolor = blue, urlcolor=blue, filecolor=green, linkcolor=red, plainpages=false}
%%%%%%%%%%%%%%%%%%%%%%%%%%%%%%%%%%%%%%%%%%%%%%%%%%%%%%%%%%%%%%%%%%%%%%%
% \pagestyle{fancy}

% Top and Bottom Line Rules
% \renewcommand{\headrulewidth}{0.4pt} %0.4pt
% \renewcommand{\footrulewidth}{0.4pt}
% \fancyheadoffset{9pt}
% \fancyfootoffset{9pt}
% Line spacing
\renewcommand{\baselinestretch}{1.1} %1.5
%%%%%%%%%%%%%%%%%%%%%%%%%%%%%%%%%%%%%%%%%%%%%%%%%%%%%%%%%%%%%%%%%%%%%%%
\date{}

\title{ \textbf{Applications of Cellular Automata: Wildfire Spread simulation} \\  }

\author{\\ \Large{Thomas Barratt, Jacob Dalrymple  } }
\date{\today}
%%%%%%%%%%%%%%%%%%%%%%%%%%%%%%%%%%%%%%%%%%%%%%%%%%%%%%%%%%%%%%%%%%%%%%%
% \pagenumbering{roman}
\begin{document}
\maketitle

\newpage

\section{Abstract}

\section{Introduction}
By implementing a 6-state based model of fire spread across a variable-height terrain with wind and areas of varying flammability, we have been able to simulate the spread of real forest fire across a simulated environment, dictated by the specification given to us.

By using a minimal state set and an auxiliary 20-value cell attribute grid, we can model complex relationships between the properties of an area in a forest fire to create a stochastic state transition function for each cell in the next time step.

Attempting to relate our simulation to the real world values of forest fires across a similar terrain, we have gradually improved the weighting of various factors to generate a suitable fire spread. 

\section{Introduction and Background: Literature Review}
% \begin{displayquote}
%   Your brief restatement of the problem
%   that you have been asked to solve. Carry out a search of the literature to identify at least two
%   examples of journal papers that describe CA-based models relevant to this system. What features
%   did they model? What did they predict?
% \end{displayquote}
Consulting existing research into the field of cellular automata can give us a valuable insight into how best to proceed when considering the formulation of a model which can describe the spread of wildfire. Considering what properties are being modelled and the ways in which transitions between state are decided can give us a grounding in this specific problem area. Research into the real world applications of forest fires are fruitful and well explored: using CA to explore forest fires is something that is well researched [\cite{ntinas2017parallel}
\cite{clarke1994cellular}
\cite{trunfio2011new}].

Considering the evidence existing papers have used to justify their results, including their empirical values formulated by authors familiar with the problem domain, can provide a good metric for the accuracy of our results. 

% Because this project seeks to show the effect of wildfires in relation to a municipality and surrounding area,the real world implications are obvious, and should be considered throughout our research into this domain.

There are two notable papers in this field whose outcomes seem to overlap strongly with our intentions. Both of these papers consider simulating real world forest fires using cellular automata, but have different approaches to the use of neighbourhoods and states, and their relationship with the real world properties that affect forest fires.


\cite{HERNANDEZENCINAS20071213} seeks to model forest fires using a hexagonal grid, modeling various different fuel values as states. By using a hexagonal model, the size of each cells' neighbourhood is increased - allowing more complex calculations to be derived on the basis of a `near neighbourhood' and `distant neighbourhood'.
The authors of the paper opt to use 3 properties when considering the flammability rate of a cell: wind, topography, and the rate of fire speed. This simplistic model differs from that of \cite{ALEXANDRIDIS2008191}, which includes 
spread and shape of a forest fire front; the fuel type (type of vegetation); humidity; wind direction and magnitude; terrain topography (slope and natural barriers), fuel continuity (vegetation thickness); and spotting - a phenomenon where burning material is transferred by the wind or other reasons such as the fling of flaming pinecones to areas that are not adjacent to the fire front.

Although the inclusion of additional parameters in \cite{ALEXANDRIDIS2008191} allows an implementation with greater accuracy in relation to real forest fires, this is only true if the correct relationships and weightings are defined.

    % \begin{itemize}
    %   \item Wind
    %   \item Topography
    %   \item Rate of fire spread
    % \end{itemize}

  \cite{HERNANDEZENCINAS20071213} following states are defined in the paper: \\ $\left\{ 0, 0.1, 0.2, 0.3, 0.4, 0.5, 0.6, 0.7, 0.8, 0.9, 1  \right\}$, where each value represents a fuel value. This differs from \cite{ALEXANDRIDIS2008191}, where four states were used ($ \left\{ 0, 1, 2,3 \right\}$), which represent differing functional states. This choice of state representation is based on the paper's approach to the functional implementation of what means to `be on fire'.

 \cite{ALEXANDRIDIS2008191}'s model was designed to include the most impactful properties when considering the spread of wildfire. By comparing their results against that of the 1990 Spetses island wildfires, and iteratively changing constants in their transition functions, the authors simulated forest fire spread to a high degree of accuracy - their final results occupying $5.4km^2$, with the actual fire occupying $5.9km^2$.

 To calculate the chance of a cell being `on fire', \cite{HERNANDEZENCINAS20071213} uses the following approach. Each parameter [listed above] is given a weighting and a probablistic model is used for each parameter to calculate the total chance of a cell being set on fire in the next time step ($t+1$).

 The chance of fire, $p_{burn}$, is given as:

 \[ p_{burn} = p_h(1+p_{veg})(1+p_{den})p_wp_s \]
 where: \\
 $p_h = $ constant probability that a cell adjacent to a burning cell containing a given type of vegetation and density will catch fire at the next time step under no wind and flat terrain \\
 $p_{den}, p_{veg}, p_w, p_s$ = the density of vegetation, the type of vegetation, the wind speed and the slope, respectively.

 In contrast, \cite{HERNANDEZENCINAS20071213} looked to find the optimal values in any forest fire, using purely mathematical models which consider heavily the boundary conditions of each of the hexagon cells in an inhomogeneous forest neighborhood.

 This difference in approach, applied versus maximum mathematical, shows the effect of the fuzzy nature of real life. \cite{ALEXANDRIDIS2008191}, which looked towards a case study to generate its values, relied on empirical constants, whereas \cite{HERNANDEZENCINAS20071213} focused entirely on a formulaic approach.

 When considering the use of CA in our study, the question of whether a case study based approach has to be answered. Due to the greater knowledge of the existing authors on the topic, ensuring the work presented here is the best position to build on their findings is essential.

  % \cite{ALEXANDRIDIS2008191}  defines the following relationship between the neighborhood of a cell and its state: If all cells in the neighborhood of $O = (a, b)$ are unburned at time $t$ except only one adjacent cell, which is completely burned out, then at time $t + 1$, the cell $(a, b)$ is completely burned out.
 

  
    % The authors states several key criteria which affect the spread of wildfire across a terrain.
    

  
  % The CA has four necessary states:
  
  % $State = 1$: The cell contains no forest fuel. This state may describe the cells corresponding to parts of the city with no vegetation, rural areas with no vegetation etc. We assume that cells that are in this state cannot be burned.
  
  % $State = 2$: The cell contains forest fuel that has not ignited.
  
  % $State = 3$: The cell contains forest fuel that is burning.
  
  % $State = 4$: The cell contained forest fuel that has been burned down.

  The use of additional parameters such as humidity to predict the spread of the forest fire, along with their use of real world GIS values gives rise to a greater accuracy of results (in relation to the real world spread of forest fires) when compared to the default criteria to be implemented.

  To meet the criteria given, the insights regarding the weighting of different parameters and the use of a probabilistic model to calculate the chance of a cell burning offer a credible starting point.

  The two papers discussed above implement the idea of 'fuel' in different ways. \cite{ALEXANDRIDIS2008191} structures fuel as a property of the states, whereas \cite{HERNANDEZENCINAS20071213} provides different states for different levels of fuel. 
  
  The advantage of moving the fuel state into an attribute of the cell is the greater precision of fuel that can be stored, while also maintaining a smaller amount of states. By limiting fuel state to discrete variables, the forest fire spread will lose valuable resolution which can be maintained with a different data structure.

\section{Methodology}
\begin{displayquote}
  - A short summary (IN YOUR OWN WORDS) of how the CA approach can
  be applied to model a forest fire. This should assume that the reader is a non-expert in modelling or
  Computer Science in general. You should explain in English (as opposed to simply using code), how
  you have extended the model you have been given in order to investigate the features mentioned.
  You can also use simple flow or state transition diagrams to support you description. You can also
  refer to relevant python code included in an appendix.
  It is expected that in extending/developing your model you will have to make some assumptions
  about how to implement particular behaviours, and also in terms of the parameters (values) you
  choose to use e.g. to represent different fuel resources/burning times. You are not expected to
  become experts in this area but you should at state justify any assumptions you make.
\end{displayquote} 

Cellular Automata is a term to describe the simulation of a discrete number of cells and interactions across a cell space (grid). Each cell can have one state at any one time step (generation). Cells can change state. The permitted changes from any one state to another state is determined by a transition function. The transition function can consider global properties, such as the current time step, as well as local properties, such as the cell's current state, as well as its neighbourhood - a set of cells near the cell. 

By representing small sections of the terrain as cells, cellular automata can be used to simulate the spread of the fire across the grid. The amount of cells can be thought of as the resolution of the simulation. 

After consulting existing literature, the following functional states were devised. 

\begin{enumerate}
  \setcounter{enumi}{-1}
  \item \textbf{Burnt out} - cells which were once on fire, but have no fuel remaining.
  \item \textbf{Burnable grass} - the default state for the terrain, as listed by the specification
  \item \textbf{Dense Forest} - Thick forest with more `fuel' than burnable grass, but which is harder to ignite.
  \item \textbf{High Flammable Scrub} - A low fuel but highly flammable substance, found in the valley in the specification
  \item \textbf{On fire} - cells which are on fire
  \item \textbf{Buildings} - cells which are buildings, this represents the town on the map.
  \item \textbf{Water} - a state which can never be on fire.
\end{enumerate}

These states represent the same approach as that of \cite{ALEXANDRIDIS2008191}, where each state represents a functionally different cell. This state model is an extension of a binary state CA given as a starting point for this assignment. The migration away from a two state model was necessary to capture the detail provided in the specification and also to implement some of the core findings from existing literature.

Burnable grass, Dense Forest and High Flammable Shrub are all variations of a burnable substance, and although this could be managed entirely within the attribute grid for each cell, having dedicated states allows the representation of the differing areas with different colours with the CaPyle simulation engine \cite{capyle_home_2016}. 

\begin{figure}[h]
  \centering
    \begin{tabular}{ || l | l ||}
      \hline
      \textbf{State}            &  \textbf{Valid transition states}  \\ 
      \hline
      Burnt Out        &  Burnt out           \\       
      Burnable grass         &  Burnable grass, On fire   \\
      On fire          &  On fire, Burnt out  \\   
      Buildings          &  Buildings, On fire  \\   
      Water     &  Water        \\
      \hline
 
    \end{tabular}
  \caption{Valid state transitions for each of the states in the CA.}
  \label{Valid state transitions}
\end{figure}

To produce a performant simulation which can cope with many generations, values which can be precomputed are done so in a setup function, and are then added to the attributes grid as an additional dimension.

The transitions between states are then calculated using these attributes for each cell (along with the local properties of the cell, such as number of neighbours in a certain state), which are stored in an $20$-value array for each cell in a grid equal in width and height to the grid. 

\begin{itemize}
  \item [0.] Height - Scalar value (m)
  \item [1.] Flammability
  \item [2.] Humidity
  \item [3.] Fuel
  \item [4-12.] Wind difference vectors (difference between wind in cell and neighbours) ($ms^{-1}$)
  % \item 8 Wind vectors 
  % \item 8 Height vectors  - assumption about heights
  % \item Flammability
  % \item Humidity
  % \item Fuel
\end{itemize}

\subsection{Modelling the effect of wind in the model}
Wind has a significant effect on wildfires, with research suggesting that even a wind speed of $2$ to $6ms^{-1}$ can increase the spread of the fire by an additional 50\% \cite{Beer1991}. As wind heavily is discussed in both \cite{ALEXANDRIDIS2008191} and \cite{HERNANDEZENCINAS20071213}, there is a need to factor wind into this model.

By computing wind vectors for the whole grid at setup time, we can then use the difference in wind angle and `fire angle` to predict the effect that the wind will have on the fire, and consequently, the resultant directional changes.

The fire angle can be defined as.

By initially setting wind to be two scalars, an  $\pm$ value for wind in the X axis, and in the Y axis, and then converting this to a vector, we can then use $acos(v_1,v_2)$ where our wind vector $= v_1$, and our `fire angle' $= v_2$, to calculate the possible `magnitude` of fire for each neighbour of each cell. As each cell has 8 neighbours, an 8-value array is generated, and added to our cell attributes grid.

\subsubsection{Modelling the effect of height in the model}
The implementation of height in this model begins with an assumption: rate of flammability is inversely proportional to the height difference.

\subsection{Calculating transitions}

Using the methods defined in \cite{ALEXANDRIDIS2008191} as a basis for our transitions, our grid is computed at every interval using two behaviours, \code{ignite} and \code{reduce\_fuel}, which perform as their names suggest, both returning new, computed versions of the grid.

\subsubsection{Ignite}
% Height

% Pass windx, windy, scalars, create vector from both  (Constant across all cells - vector), calculate magnitude of vector.
% Fire vectors calculated in neighbours N, NE, E....
% define angle array.
% Calculate difference in angle between the two.
% Supply wind vector and fire vector, calculate the angle difference, results in 8 different angle diffs, for each neighbour

% Uses formula to calculate the angular differences, and the wind vectors, using formula defined in \cite{ALEXANDRIDIS2008191}, we can plug in values:

% Weightings used in the paper.

% Result is wind weighting across all 8 different neighbours.

% Calculate wind weight, initially 0, (8,200,200), wind weight increased by number of neighbours on fire.

Base probability 0.5, normal woodland could catch on fire even if no wind as long as one neighbour is on fire. 

Given the parameters, a probability is derived: the chance of the current cell setting on fire in the next time step. This is consistent with the method in \cite{ALEXANDRIDIS2008191}.

If at least one cell is on fire in the cell's neighbourhood, the chance of catching fire is calculated by the using a weighted sum of probabilities, dependent on the wind magnitude, the flammability of the cell that is on fire, and the difference in height compared to the neighbourhood.

Generate random array, see if below it, schotachisticty.


How long does a normal forest fire take to travel 5km, then divide by 2: that's our speed. 


This can be represented as the following equation:
\subsubsection{Reduce Fuel}
The fuel reduction technique considers every cell in the grid whose fuel attribute is greater than 0. The algorithm subtracts the \code{rate\_of\_flam} attribute from the current \code{fuel} value, clipping the results to 0. 

Because the \code{rate\_of\_flam} attribute is provided for each cell at setup, representing the average loss of fuel per time step, tweaking the fuel values to closely mimic their real world counter parts is essential.

\section{Results}
\begin{displayquote}
  In this section you should describe what simulations you
carried out under which conditions (e.g. parameter sets mentioned above, wind direction,…) and
how long the simulation was run for. You should use fully labelled diagrams (e.g. screenshots or
where appropriate, line graphs) to display your results and you should also describe your results in
text. Additional results can be included in an appendix
\end{displayquote}
  \subsection{Dense forest locations}
  \subsection{Best water locations}
  \subsection{Wind Direction - East to West}
  \subsection{Wind Direction - North to South}
  \subsection{Wind Direction - North East}
  \subsection{Wind Direction - South West}
  \subsection{Resolution     -  50 x  50}
  \subsection{Resolution     - 200 x 200}
  \subsection{Resolution     - 400 x 400}
\newpage
\section{Discussion of model and conclusions} 
\section{Conclusion}

\newpage
\renewcommand{\bibname}{Bibliography}
\bibliographystyle{apacite}
\bibliography{Bibliography.bib}


\end{document}